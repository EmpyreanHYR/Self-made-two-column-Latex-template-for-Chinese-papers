\documentclass[twocolumn]{article}
\usepackage{xeCJK} % 支持中文
\usepackage[left=2cm,right=2cm,top=2cm,bottom=2cm]{geometry} % 设置页边距
\usepackage{lipsum} % 生成示例文本
\usepackage{abstract} % 定制摘要

\usepackage{indentfirst} % 设置首行缩进
\setCJKmainfont{SimSun} % 设置中文字体为宋体
\setmainfont{Times New Roman} % 设置英文字体


% =================================
% 数学与公式
% =================================
\usepackage{amsmath}        % AMS 提供的数学公式增强宏包,提供 align, gather 等环境
\usepackage{amssymb}        % 包含大量数学符号,如 R(实数集)、\lesssim 等
\usepackage{amsfonts}       % AMS 提供的数学字体,用于特殊字体(如花体字母)
\usepackage{bm}             % 使公式中的粗体符号(如向量)显示为直立粗体 (\bm)

% =================================
% 图形、表格与浮动体
% =================================
\usepackage{graphicx}       % 引入图片(使用 \includegraphics 命令)
\usepackage{float}          % 增强对浮动体(图、表)的控制,例如 H 定位符
\usepackage{booktabs}       % 绘制专业、美观的三线表(使用 \toprule, \midrule, \bottomrule)
\usepackage{multirow}       % 表格中合并行
\usepackage{makecell}       % 表格中换行和调整单元格边距
\usepackage{caption}        % 更好地自定义图表标题格式
\usepackage{subcaption}     % 在一个浮动体中并排放置多个子图或子表


% =================================
% 链接与排版增强
% =================================
\usepackage{hyperref}       % 生成可点击的超链接(目录、引用、参考文献等),通常应最后加载
\hypersetup{
	colorlinks=true,        % 彩色链接
	linkcolor=black,        % 内部链接颜色
	citecolor=blue,         % 引用链接颜色
	urlcolor=blue           % URL 链接颜色
}
\usepackage{enumerate}      % 增强列表环境(itemize, enumerate),允许自定义标签格式

% =================================
% 算法
% =================================
% 推荐使用下列其中一个宏包:
\usepackage{algorithm}      % 算法浮动体环境
\usepackage{algpseudocode}  % 伪代码排版宏包(与 algorithm 配合使用)
% 或
% \usepackage[linesnumbered, ruled]{algorithm2e} % 另一种流行的算法排版宏包






% 取消摘要环境自带的标题
\renewcommand{\abstractname}{}
\renewcommand{\absnamepos}{empty}

\title{
	中文标题 \\
	\vspace{0.5em} \normalsize 中文作者姓名 \\
	{\small 中文作者单位,单位地址,邮编} \\
	\vspace{1em} % 英文标题和中文标题之间的空间
	\large Example of Chinese Title \\
	\vspace{0.5em} \normalsize English Author Name \\
	{\small English Author Affiliation, Address, Zip Code}
}
\date{} % 不显示日期

% 将参考文献标题改为“参考文献”
\renewcommand{\refname}{参考文献}
\begin{document}
	
	\twocolumn[
	\maketitle
	\begin{minipage}{\textwidth} % 使用minipage使摘要和关键词保持单栏格式
		\centering % 使得摘要标题居中
		{\noindent \textbf{\large 摘要}} % 中文摘要标题
		\vspace{0.5em} % 添加一些垂直空间
	\end{minipage}
	
	\begin{minipage}{0.9\textwidth} % 控制摘要正文宽度
		这里是中文摘要内容。
	\end{minipage}
	
	\vspace{1em} % 在摘要和关键词之间添加一些空间
	{\noindent \textbf{关键词:}关键词1, 关键词2, 关键词3}
	\vspace{1em} % 在中文摘要和英文摘要之间添加一些空间
	
	\begin{minipage}{\textwidth}
		\centering
		{\noindent \textbf{\large Abstract}} % 英文摘要标题
		\vspace{0.5em} % 添加一些垂直空间
	\end{minipage}
	
	\begin{minipage}{0.9\textwidth} % 控制摘要正文宽度
		This is the content of the English abstract.
	\end{minipage}
	
	\vspace{1em} % 在摘要和关键词之间添加一些空间
	{\noindent \textbf{Keywords:} Keyword1, Keyword2, Keyword3}
	\vspace{1cm} % 在摘要和正文之间添加一些空间
	]
	
	
	
	
	% 正文内容
	\section{引言}
	
	这是引言部分。
	
	\section{相关工作}
	这是第一段。
	
	这是第二段
	
	\section{方法}
	\lipsum[4-5]
	
	\section{实验}
	\lipsum[6-7]
	
	\section{结论}
	\lipsum[8]
	
% 使用 thebibliography 环境格式化参考文献列表
\begin{thebibliography}{99} % 99 是最大编号的宽度
	\bibitem{ref1} 参考文献1的详细信息.
	\bibitem{ref2} 参考文献2的详细信息.
	% 继续添加更多参考文献...
\end{thebibliography}
	
\end{document}
