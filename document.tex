\documentclass[twocolumn]{article}
\usepackage{xeCJK} % 支持中文
\usepackage[left=2cm,right=2cm,top=2cm,bottom=2cm]{geometry} % 设置页边距
\usepackage{lipsum} % 生成示例文本
\usepackage{abstract} % 定制摘要

\usepackage{indentfirst} % 设置首行缩进

\setCJKmainfont{SimSun}[
BoldFont = SimHei, 
AutoFakeBold = {2.0} % 如果 SimHei 仍然找不到,尝试自动加粗
]
\setmainfont{Times New Roman} % 设置英文字体


% =================================
% 数学与公式
% =================================
\usepackage{amsmath}        % AMS 提供的数学公式增强宏包,提供 align, gather 等环境
\usepackage{amssymb}        % 包含大量数学符号,如 R(实数集)、\lesssim 等
\usepackage{amsfonts}       % AMS 提供的数学字体,用于特殊字体(如花体字母)
\usepackage{bm}             % 使公式中的粗体符号(如向量)显示为直立粗体 (\bm)

% =================================
% 图形、表格与浮动体
% =================================
\usepackage{graphicx}       % 引入图片(使用 \includegraphics 命令)
\usepackage{float}          % 增强对浮动体(图、表)的控制,例如 H 定位符
\usepackage{booktabs}       % 绘制专业、美观的三线表(使用 \toprule, \midrule, \bottomrule)
\usepackage{multirow}       % 表格中合并行
\usepackage{makecell}       % 表格中换行和调整单元格边距
\usepackage{caption}        % 更好地自定义图表标题格式
\usepackage{subcaption}     % 在一个浮动体中并排放置多个子图或子表


% =================================
% 链接与排版增强
% =================================
\usepackage{hyperref}       % 生成可点击的超链接(目录、引用、参考文献等),通常应最后加载
\hypersetup{
	colorlinks=true,        % 彩色链接
	linkcolor=black,        % 内部链接颜色
	citecolor=blue,         % 引用链接颜色
	urlcolor=blue           % URL 链接颜色
}
\usepackage{enumerate}      % 增强列表环境(itemize, enumerate),允许自定义标签格式

% =================================
% 算法
% =================================
% 推荐使用下列其中一个宏包:
\usepackage{algorithm}      % 算法浮动体环境
\usepackage{algpseudocode}  % 伪代码排版宏包(与 algorithm 配合使用)
% 或
% \usepackage[linesnumbered, ruled]{algorithm2e} % 另一种流行的算法排版宏包


\usepackage{caption} % 用于定制标题格式
\usepackage{xpatch}  % 用于修改 \captionname 宏(可选,但推荐用于中文/双语环境)

% 1. 定义一个命令来存储中文的“图”和“表”名称,以配合 xeCJK 的中文环境
\renewcommand{\figurename}{图}
\renewcommand{\tablename}{表}

% 2. 使用 caption 宏包定制标题格式:居中对齐,并在编号和文字之间添加空格
\captionsetup{
	justification=centering,   % 居中对齐
	labelsep=quad,             % 标签和文字之间的间距(quad 略大于 space)
	font=normal,               % 字体设置为正常
}


\usepackage{cleveref} % 交叉引用
\crefname{figure}{图}{图}
\Crefname{figure}{图}{图}
\crefname{table}{表}{表}
\Crefname{table}{表}{表}
\crefname{section}{节}{节}
\Crefname{section}{节}{节}
\crefname{equation}{公式}{公式}
\Crefname{equation}{公式}{公式}


\usepackage{cite} % 无自动上标,需要手动上标,下方才用重新定义实现
% =================================
% 重新定义 \cite 命令为上标(带方括号)
% =================================
% 1. 备份原始的 \cite 命令
\let\oldcite\cite 
% 2. 重新定义 \cite 命令,将其内容放入 \textsuperscript{}
\renewcommand*\cite[1]{\textsuperscript{\oldcite{#1}}}

%\usepackage[numbers, super]{natbib} % 无方括号,与cite选用

\usepackage{fancyhdr} % 页眉页脚


% 取消摘要环境自带的标题
\renewcommand{\abstractname}{}
\renewcommand{\absnamepos}{empty}

\title{
	\textbf{中文标题} \\
	\vspace{0.5em} \normalsize 中文作者姓名 \\
	{\small 中文作者单位,单位地址,邮编} \\
	\vspace{1em} % 英文标题和中文标题之间的空间
	\large \textbf{Example of Chinese Title} \\
	\vspace{0.5em} \normalsize English Author Name \\
	{\small English Author Affiliation, Address, Zip Code}
}
\date{} % 不显示日期

% 将参考文献标题改为“参考文献”
\renewcommand{\refname}{参考文献}
\begin{document}
	
	\twocolumn[
	\maketitle
	\begin{minipage}{\textwidth} % 使用minipage使摘要和关键词保持单栏格式
		\centering % 使得摘要标题居中
		{\noindent \textbf{\large 摘要}} % 中文摘要标题
		\vspace{0.5em} % 添加一些垂直空间
	\end{minipage}
	
	\begin{minipage}{1\textwidth} % 控制摘要正文宽度
	
	% 临时设置 \parindent 的值为 2em(中文两字符缩进)
	\setlength{\parindent}{2em}
	
	\indent	这里是中文摘要内容。这里是中文摘要内容。这里是中文摘要内容。这里是中文摘要内容。这里是中文摘要内容。这里是中文摘要内容。这里是中文摘要内容。这里是中文摘要内容。这里是中文摘要内容。这里是中文摘要内容。这里是中文摘要内容。这里是中文摘要内容。这里是中文摘要内容。这里是中文摘要内容。这里是中文摘要内容。这里是中文摘要内容。这里是中文摘要内容。这里是中文摘要内容。
	\end{minipage}
	
	\vspace{1em} % 在摘要和关键词之间添加一些空间
	{\noindent \textbf{关键词:}关键词1, 关键词2, 关键词3}
	\vspace{1em} % 在中文摘要和英文摘要之间添加一些空间
	
	\begin{minipage}{\textwidth}
		\centering
		{\noindent \textbf{\large Abstract}} % 英文摘要标题
		\vspace{0.5em} % 添加一些垂直空间
	\end{minipage}
	
	\begin{minipage}{1\textwidth} % 控制摘要正文宽度
	\setlength{\parindent}{2em}
	
	\indent	This is the content of the English abstract. This is the content of the English abstract. This is the content of the English abstract. This is the content of the English abstract. This is the content of the English abstract. This is the content of the English abstract. This is the content of the English abstract. 
	\end{minipage}
	
	\vspace{1em} % 在摘要和关键词之间添加一些空间
	{\noindent \textbf{Keywords:} Keyword1, Keyword2, Keyword3}
	\vspace{1cm} % 在摘要和正文之间添加一些空间
	]
	
	
	
	
	% 正文内容
% =================================
%	使用 \input 导入短小的内容(如图片列表、表格或自定义命令集合)。
%	
%	使用 \include 导入完整的章节(如 \section{...} 及其内容),因为它具有分页和有选择地编译的优势。
% =================================
	
	\section{引言}

这是引言部分。
	
	\input{texfiles/2.相关工作.tex}
	
	\section{方法}
\lipsum[4-5]
	
	\section{实验}
\lipsum[6-7]	
	
	\section{结论}
\lipsum[8]



这是一个标签引用\cref{fig:lavender-96808701280}。

\begin{figure}
	\centering
	\includegraphics[width=0.7\linewidth]{fig/lavender-9680870_1280}
	\caption{花}
	\label{fig:lavender-96808701280}
\end{figure}



这是一个参考文献引用\cite{ref1}。

这是两个参考文献引用\cite{ref1,ref2}。

这是多个参考文献引用\cite{ref1,ref3,ref4,ref5}。	
	
	
	
	
	
% 使用 thebibliography 环境格式化参考文献列表
\begin{thebibliography}{99} % 99 是最大编号的宽度
	\bibitem{ref1} 参考文献1的详细信息.
	\bibitem{ref2} 参考文献2的详细信息.
	\bibitem{ref3} 参考文献3的详细信息.
	\bibitem{ref4} 参考文献4的详细信息.
	\bibitem{ref5} 参考文献5的详细信息.
	% 继续添加更多参考文献...
\end{thebibliography}
	
\end{document}
